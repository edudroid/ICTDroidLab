\documentclass[conference,letterpaper]{IEEEtran}

\usepackage[english]{babel}


\usepackage[dvips]{graphicx}

\usepackage{graphicx}


\usepackage{longtable}







\graphicspath{{./figs/}}
%%\usepackage[T1]{fontenc}
%%\usepackage[latin2]{inputenc}
%%\usepackage{t1enc}
\usepackage{epsfig}

\begin{document}
\selectlanguage{english}

\title{DroidLab: mobile sensing made easy, fast and cheap}
\author{\IEEEauthorblockN{Bal\'azs Lajtha, Rolland Vida}
  \IEEEauthorblockA{Department of Telecommunications and Media Informatics\\
    Budapest University of Technology and Economics\\
     % Magyar tud\'osok k\"or\'utja 2., Budapest, Hungary, H-1117  \\
    Email: \{lajtha.balazs, vida\}@tmit.bme.hu}}

%\maketitle

% \long\def\symbolfootnotetext[#1]#2{\begingroup  \def\thefootnote{\fnsymbol{footnote}}\footnotetext[#1]{#2}\endgroup}

% \symbolfootnotetext[0]{The second author was supported by the Janos Bolyai
  % Fellowship of the Hungarian Academy of Sciences.}

\maketitle


\begin{abstract}
We are living in a parallelized world. What was thought to be impossible for one can now easily be achieved by many. On one hand, we have the cloud: an army of obeying faceless machines. On the other hand we have the crowd: A colorful collection of volunteers. Clouds are built and sold but to control the crowd, each member of has to be won over.
\end{abstract}

\begin{IEEEkeywords}
\end{IEEEkeywords}
 
\section{Introduction}
\label{sec:intro}
"Crowdsourcing is the act of taking a job traditionally performed by a designated agent (usually an employee) and outsourcing it to an undefined, generally large group of people in the form of an open call." This was the initial definition of crowd sourcing when first used in 2006  by Jeff Howe and Mark Robinson. Two main orthogonal problem sets can be solved using crowd sourcing: data gathering and data processing. When a data gathering problem is distributed among members of the crowd we use the term crowd sensing or participatory sensing, while for distributed computational tasks we use the general term crowd sourcing.

Another way to categorize crowdsourced applications is weather the contributors profit directly of the gathered or processed information. Wikipedia uses the power of the crowd to provide a service that can directly be used by the contributors. When it comes to crowd sourcing platforms, creating a framework to acquire information is much easier than creating a generic framework that is able to present users with the results. In many cases the beneficiaries of the crowd sourced tasks are different than those providing the data. DroidLab targets these scenarios.

Probably the most popular crowd sourced computational problem was Seti@Home, a project lasting several years and aimed to ??? starts and stuff, extraterestrial ???. This problem was divided into a large number of subtasks and distributed among the willing participants. Users could set a Seti screen-saver that download and solved these subtasks when the user was away. Seti's success came from it's innovative design and from the interesting task that it performed. From the point of view of the volunteer, this was a low cost, low gain activity. PCs back than had long startup times and high standby power consumption, once Seti was installed it didn't pose a visible overhead, and provided the user with insight of the performed work. Some early gamification elements were also implemented like leaderboards.

A somewhat different example is the Google Image classification game. In this game two players cooperated to solve a simple puzzle. An image was displayed in both player's browsers. The first player had to find five words that described the image. Then the second player had to guess those words. This human image processing game allowed Google to improve Image Search results. The players were motivated to play the game for the game itself, and for added gamification elements.

Both examples involved utilization of computational resources owned by the users to perform a highly parallelized task that the users perceived as common good, but didn't benefit directly from the results of their work. In both cases, users received an input that they processed based on a given algorithm. The tasks were not personalized. Seti@Home's tasks were solely performed by a machine, while Google Image games had to be solved by human users. In both examples users had extrinsic motivators.

Participatory sensing involves data collection outside of the system. Sensing can be automated or can involve the active participation of the user. Before smartphones, sensing was either done through expensive Internet-connected sensors or with the participation of the user through fixed personal computers. Sensors had to be deployed, and sensing involving the user had large delay and required considerable effort from the part of the user. The evolution of smartphones eased many new complex sensing applications. Smartphones have the connectivity that enable real time, high bandwidth communication. The devices can access precise location data, and are equipped with a wide range of sensors from cameras and microphones to accelerometers, gyroscopes and barometers. New mobile operating systems offer many seamless ways for the sensing system to prompt for user input.

To exploit these possibilities we offer DroidLab. DroidLab is a cloud backed mobile framework still under development, that aims to solve key challenges of mobile crowd sourcing. In this article we will present successful and promising mobile crowd sourcing application and crowd sourcing frameworks. We will identify the key challenges of mobile crowd sourcing and define the use cases for our framework. Then present the architecture and design of our solution and conclude with the evaluation of the current framework and define the future works.
\section{Related work}
\label{sec:related_work}
We already mentioned two successful examples of crowd sourcing. Seti@Home and Google Image Game both targeted desktop platforms. Seti@Home was an automated solution while Google Image Game involved user interaction.

While it's easy to find proposed applications for participatory sensing, academic articles discussing the challenges of such applications and their effects on society, there is merely a couple of successfully deployed applications. Mobile broadband penetration is 74.8 in the developed world while 19.8 in the developing world. While this equals in a larger number of subscribers in the developing regions, most application downloads come from the developed countries. Hence, most crowd sourced applications address "first world problems".

\subsection{Transportation}
Getting from point A to point B raises many issues, some of them can be solved by a crowd sourced approach. When using a car, knowing the traffic conditions in advance gives a huge advantage. Waze is one of the many traffic aware navigation applications that uses phone's sensors to identify traffic conditions. Besides traffic, road quality is an important factor too, both for drivers and cyclists. Smartphone acceleration sensors can accurately detect potholes. While no single pothole detection application has emerged, several publications discuss such a framework.

When arriving to the destination finding a parking spot will become the main concern. Google launched Open Spot, an experimental application that relied on user interaction to detect freed up parking spaces. The application was a failure, the problem remains unsolved.

Public transportation can benefit of crowd sensed information. HotStop Live is an application that provides live transit information gathered by the users.

\subsection{Presence and phone book}
The user's presence can be detected using sensor information, enabling a framework to communicate detailed information about a contact's state before calling her. Phone books can be shared among users to ease reverse lookup of phone numbers. Mr Number helps it's users avoid unwanted calls by collaboratively maintaining a black list of telemarketers and robocallers.

\subsection{Weather and environment}
The authors have participated in the development of the mobile application of Id\"ok\'ep, a Hungarian weather service. Id\"ok\'ep had more than a hundred deployed online weather stations and the mobile application enabled users to share weather photos, perceived weather condition and analogue measurement results.

While smartphones are not equipped with the proper sensors to become a weather station, their near field communication technologies (Bluetooth, WiFi, NFC) enable them to be extended with peripherals capable of measuring air quality, humidity or temperature. In the  Common Sense project bluetooth enabled air pollution sensors were deployed with the mobile application to measure and report air pollutants. Microphones enable smart phones to capture noise level and process noise pollution.

CreekWatch is an application that helps monitor water quality and pollution relying on user's reports and photos.

\subsection{Network coverage}
Network access and quality is an important information for smartphone users. Network information can be sensed through available network status information and active probing. The two main channels are WiFi networks and cellular networks. WiFi Finder maintains a database of opened WiFi networks. It's built and updated based on automated reports of opened WiFi hotspots. It provides a map of the active free WiFi hotspots and their properties.

RootMetrics monitors cellular coverage, it let's users share their active measurement data to process a coverage map.

\subsection{Lifestyle and health}
Healthcare in modern societies is a main concern. Smartphones' sensors can help on many different levels. For individuals it can help to monitor activity, track exercises, with external Bluetooth devices log blood pressure and other metrics. In a family results of the elderly members can trigger alarms at the younger generations. Same can be implemented in a smaller community. On national scale data provides insight of the health level of the people. On a global scale these statistics can help find correlation between lifestyle and health.

A crowd sourced ad-hoc communication network can be useful both on outdoor events where cellular coverage is not designed for many users, and in case of an outage to help first responders contact those without mobile service. Crowd sensing is useful to detect and track parades, estimate number of participants.

\subsection{Frameworks}
Several frameworks have been proposed for different aspects of mobile crowd sourcing, most of them experimental. mCrowd is a mobile crowdsourcing platform that enables iPhone users to post or tag images or answer questions. mCrowd was a prototype application that demonstrated a micro-payment system rewarding the contributors.
mClerk is a platform that distributes tasks and enables workers to post their responses. mClerk targeted third world users without smartphones. The key challenge of mClerk was to create a communication protocol that enables data transfer over text messages, enabling traditional phone users to receive complex tasks and report results. Both platforms focus on the active contribution of the users.

Other frameworks target the communication protocols used in crowd source applications. When information is both gathered and consumed by the same crowd, it is crucial to have a robust, multi-source, multi-destination multicast infrastructure. Fan-Bai et al proposed such a system for vehicular networks. Demirbas et al developed an overlay that uses Twitter to distribute sensor information. Szabo et al investigated the use of XMPP as a carrier protocol for crowdsourced applications.

MEDUSA is an experimental framework that enables the execution of sensing tasks on a swarm of devices connected to a cloud based server. Ryong Ra et al created a scripting language MedScript and a prototype of the runtime. MedScript introduced an abstraction that helps describe crowdsourcing tasks along with language constructs that map to user involvement like requesting user input or user actions and rewarding users. MEDUSA also defines a software architecture and provides a reference implementation for cloud-backed crowd sourcing frameworks.

Opposed to MEDUSA DroidLab focuses on automated sensing. We designed our framework to be able to operate in the background, but taking in consideration the user's resources: our first contribution is a detailed permission and quota management system that gives full control to the user of DroidLab's activity.

Our second contribution is the plugin-based design that allows the framework to be tailored to the devices capabilities. There exist several thousand different makes of Android devices, dozens of sensors, several different network stack implementations with different feature sets, and many different type of usages. DroidLab contains different plugins for different sensors and data types. This enables the user to configure his DroidLab installation to his comfort zone. We took in account the permission system of Android, and designed the framework to require only network access in itself.

Our third contribution is the DroidLab API, a Java API that eases development of crowd sourcing application with functions to utilize the plugin's functionality and gives access to sensing related tasks like scheduling and data collection.

In the following section we will define the intended use of the framework. We will describe the challenges of such a platform. In section XXX XXX we will present the design of the platform and review the state of the art of the implementation. We will conclude by evaluating the current implementation and laying out our future work.

\section{Use cases and challenges}
\label{sec:use_cases}
Applications mentioned in the previous chapter have many properties in common:
- Only work in areas where the application's coverage is high
- Provide services to the users
- Services provided can be consumed on a mobile device through a specific application
This results in well polished, feature rich applications that emphasize the provided services. Users download the applications for their own benefit. If an application doesn't reach enough users or active user contribution is required and is too costly (see Google's parking app) the service dies out. This is a high cost, high risk scenario.






Wikipedia motivation: using results vs 
\section{Solutions}
\label{sec:solutions}
\section{Conclusions}
\label{sec:conclusion_and_future_work}
\section{Acknowledgment}
\label{sec:acknowledgment}
http://mobithinking.com/mobile-marketing-tools/latest-mobile-stats/a\#subscribers
http://au.businessinsider.com/comparing-app-downloads-across-countries-2013-6
http://www.androidauthority.com/google-labs-open-spot-a-useful-application-that-no-one-uses-15186/
http://www.waze.com/
http://live.hopstop.com/
http://www.bizjournals.com/sanjose/news/2013/05/31/whitepages-acquires-crowdsourced-spam.html
http://www.lvmt.fr/ewgt2012/compendium
http://www.communitysensing.org/
https://play.google.com/store/apps/details?id=com.jiwire.android.finder
http://www.rootmetrics.com/
\end{document}

%%% Local Variables: 
%%% mode: latex
%%% TeX-master: t
%%% End: 
